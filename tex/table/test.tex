\documentclass[12pt]{ltjsarticle}
\usepackage[a4paper, margin=25mm]{geometry}
\usepackage{amsmath, amssymb}
\usepackage{physics} 
\usepackage{siunitx} 
\usepackage{enumitem}
\usepackage{float}
\usepackage{multirow}
\usepackage{booktabs} 
\usepackage{bm}
\usepackage{graphicx}
\usepackage[hidelinks,pdfusetitle]{hyperref} 
% --- siunitxとphysicsの競合回避設定 ---
% 単位を書くときは \SI{数値}{単位} または \unit{単位} を使う
% 数式の括弧は \qty(...) を使う

\title{タイトル}
\author{202312892 八尋優太}
\date{\today}

\begin{document}

\begin{center}
  \large\textbf{タイトル}
\end{center}
\begin{flushright}
  202312892 八尋優太\
\end{flushright}
\vspace{5mm}

\begin{table}[H]
  \centering
  \begin{tabular}{ccc}
    \toprule
    $\mathrm{素子}$ & $\mathrm{エネルギー}$ & $\mathrm{温度(K)}$ \\
    \midrule
    \multirow{2}{*}{$\mathrm{MOSFET}$} & $\mathrm{-\frac{\hbar}{2m}}$ & $200$ \\
     & $\mathrm{[K]}$ & $\theta$ \\
    \bottomrule
  \end{tabular}
\end{table}

\end{document}